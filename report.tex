\documentclass[11pt,onecolumn]{article} 
\usepackage{latex8}
\bibliographystyle{latex8}
\usepackage{times}
\usepackage{graphicx}
\usepackage{wrapfig}



\begin{document}
%
% paper title
% can use linebreaks \\ within to get better formatting as desired
\title{Middleware Solutions for the Internet of Things}


% author names and affiliations
% use a multiple column layout for up to two different
% affiliations

\author{
Chinwei ``Vic'' Hu\\
{\small \textit{Department of Electrical and Computer Engineering} }\\
{\small \textit{The University of Texas at Austin}}
}


% make the title area
\maketitle


\begin{abstract}
In this paper, we briefly talk about what the Internet of Things (IoT) is, why do we care, and a few selection of the state-of-the-art middleware solutions to address some of the common problems IoT faces, such as heterogeneous coordination, context-awareness communication, resource-constrained and efficient computation. In particular, we look at Grapevine\cite{grapevine}, Gander \cite{michel2013gander}, and DAIS\cite{dais}, all developed at The University of Texas at Austin.
\end{abstract}

\Section{Introduction}
Big Data \cite{manyika2011big} and deep analytics are becoming prevalent in almost every industry, thanks to the invention of the world-wide web \cite{berners1994world} and ubiquitous web services. To make better decisions in both our personal and professional life, we need to somehow weave together the information fabric across the virtual and physical domains, which introduces an extremely complex collection of sensors, protocols, context-awareness, constrained resources, real-time responsiveness, and distributed computing problems. All of these challenges converge to the notion of the Internet of Things paradigm \cite{atzori2010internet}.

The Internet of Things can be used to monitor the behavior and conditions of persons or machines, enhance real-time situational awareness of physical environment, assist better decision making based on data analytics and visualization, automatically and optimally control self-contained complex systems, and efficiently distribute resources in a grid network \cite{chui2010internet}. Nodes interacting in such grid network could be mobile phones, tablets, sensors, actuators, or essentially anything with some kind of communication beacon such as QR code or RFID tags \cite{welbourne2009building}(Radio-Frequency Identification). Due to its pervasive presence and limitless potentials, the US National Intelligence Council includes IoT in a list of ``Disruptive Civil Technologies'' with potential impacts on US national power \cite{officialdisruptive}.

The main purpose of this paper is to study a collection of middleware solutions in addressing the aforementioned challenges in the IoT, including Grapevine \cite{grapevine}, Gander \cite{michel2013gander}, and DAIS \cite{dais}, all done at The Department of Electrical and Computer Engineering at The University of Texas at Austin. In particular, we will look at their motivations and intuitions, design patterns, system architectures, implementations, and what other improvements can be made in the trending IoT middleware solutions.

\section{System Overview}
One challenge in IoT is to address and aggregate context-awareness in surrounding neighborhood, such as traffic condition of nearby intersections, skill level and interests of players in a group, or formulating a dynamic mobile network based on communication quality and available resources. Specifically in a dynamic pervasive network, we need to identify the notion of groups based on context predicates, and to define entity and group context to assess and share them in their surrounding computing environments.  Motivated by this fundamental problem, Grapevine \cite{grapevine} addresses exactly this problem by introducing a mechanism to efficiently define and share context and group memberships in a network.

\subsection{Grapevine}

To define context summary, Grapevine uses the Bloomier filter data structure \cite{chazelle2004bloomier}, which consists of a Bloom filter cascade with $O(rn)$ space complexity and false positive rate $\epsilon \propto 2^{-r}$, where $n$ is the number of elements and $r$ is the number o bits. On the other hand, a notion of groups is defined by a constraint function $f_G$ of four types: {\em Labeled, Asymmetric, Symmetric}, and {\em Context-Defined}, which is combined with the context summary of entities to formulate expressive group membership relations.

\begin{wrapfigure}{r}{0.4\textwidth}
\vspace{-20pt}
  \begin{center}
    \includegraphics[width=0.4\textwidth]{resources/grapevine_architecture.png}
  \end{center}
\vspace{-20pt}
  \caption{Grapevine System Architecture \cite{grapevine} \label{grapevine_architecture}}
\end{wrapfigure}

As shown in Fig.~\ref{grapevine_architecture}, the {\em Context Handler} computes the context summaries with Bloomier filters to formulate groups based on the group definitions, while the {\em Context Shim} attaches the updated context summary to the outgoing packets and detaches the incoming context summary of the incoming packets, which uses the {\em interceptor} design pattern to work with the {\em Context Handler}. Furthermore, a static notion of $\tau$ is used to address the tradeoff between dissemination cost and the scope of shared context.

\begin{wrapfigure}{r}{0.4\textwidth}
  \begin{center}
    \includegraphics[width=0.4\textwidth]{resources/grapevine_implementation.png}
  \end{center}
  \vspace{-20pt}
  \caption{Information Flow in the Context Shim and Context Handler of Grapevine \cite{grapevine} \label{grapevine_flow}}
    \vspace{-30pt}
\end{wrapfigure}

In implementation, Grapevine uses Java to gain transparency in deployment platform and fine-grained control over the network layer. The information flow of Grapevine, as shown in Fig.~\ref{grapevine_flow}, illustrates how the {\em Context Handler} interacts with the application to compute group membership and context, to store a map of entity context, and to create Bloomier filters for the {\em Context Shim} to access the context summary. The performance and effectiveness of Grapevine are evaluated through a simulated network of 10-by-10 grid and the Pharos multi-robot testbed\cite{agmon2008multi}.


\subsection{Gander}
In contrast to the traditional indexing and ranking methods that require a global view of context connectivity and chronicle archives such as PageRank\cite{page1999pagerank}, the Gander search engine \cite{michel2013gander} provides an IoT perspective to the spatiotemporal queries in the {\em Personalized Networked Spaces} (PNetS). Challenges of searching in the here and now include the privacy and security of personal data, rapidly changing context and conditions, and the impracticality of centrally indexing a constantly growing amount of data. Moreover, by limiting a search within constrained neighborhood and spatiotemporal context, more efficient network and computing resources could be attained.

In order to match the {\em query resolution}, {\em query constraints} and {\em reachability} need to be defined to measure the {\em relevance metric}. Furthermore, a {\em tuple space} is used to provide the {\em global virtual data structure} \cite{picco2002global} distributed in a dynamic network. Since it's impractical to obtain global view in PNetS, query protocols have to be operated within the surrounding data. To find the most relevant query results within a constrained neighborhood of nodes, Gander {\em samples} result candidates in five styles, which are illustrated in Fig.~\ref{gander_sampling}.

\begin{figure}[h]
  \begin{center}
    \includegraphics[width=1.0\textwidth]{resources/gander_sampling.png}
  \end{center}
  \vspace{-20pt}
  \caption{Query sampling styles\cite{michel2013gander}. Dashed lines represent sent messages, and darkened nodes are those who respond to a given query. (a) Flooding (b) Random (c) Probabilistic (d) Directional (e) Regional  \label{gander_sampling}}
    \vspace{-15pt}
\end{figure}

Each of the five sampling styles demonstrates the tradeoff between quality and cost in network overhead and latency. {\em Flooding} attempts to reach every node within the network hops, {\em Random} parameterizes the likelihood of responses, {\em Probabilistic} forwards received messages based on a locality probabilistic distribution, {\em Directional} constraints the sampling within a certain heading and angle, and {\em Regional} aims at a given target area. Moreover, a secondary contextual queries are carried out from potential result's perspective, which collects local conditions without adding network overhead and delays.

\begin{wrapfigure}{r}{0.4\textwidth}
\vspace{-20pt}
  \begin{center}
    \includegraphics[width=0.4\textwidth]{resources/gander_architecture.png}
  \end{center}
  \vspace{-10pt}
  \caption{{\em myGander} Architecture \cite{michel2013gander} \label{gander_architecture}}
    \vspace{-20pt}
\end{wrapfigure}

The implementation of Gander, {\em myGander}\cite{michel2012mygander}, follows the model-view-controller architecture pattern as shown in Fig.~\ref{gander_architecture}. The model consists of a tuple space implementing the data structure, a routing table that maintains the connectivity in the network, a query processor that handles incoming queries to the tuple space, performs secondary contextual queries on the collected data and meta-data, and routes the results according to its sampling style.

To evaluate sampling protocols in a real world large scale network is extremely challenging. In practice, benchmark simulation for {\em myGander} can be done with OMNet++\cite{varga2001omnet++} for networking and SUMO\cite{krajzewicz2002sumo} for node mobility, in which each node uses its own global virtual data structure to process queries and connects to other mobile node through a proxy.

\subsection{DAIS}
In most cases, sensor nodes such as mobile phones and embedded processors have fairly limited power and computational resources, which adds complexity and barrier to the programmers. To fully facilitate application developments in the sensor network of the IoT, low-level specifications of sensor selection and communication need to be abstracted from the programmers for them to focus on high-level designs of sophisticated pervasive applications, which is the goal of DAIS (Declarative Applications in Immersive Sensor networks) \cite{dais}.

\begin{wrapfigure}{r}{0.6\textwidth}
\vspace{-30pt}
  \begin{center}
    \includegraphics[width=0.6\textwidth]{resources/dais_architecture.png}
  \end{center}
  \vspace{-20pt}
  \caption{DAIS Architecture \cite{dais} \label{dais_architecture}}
    \vspace{-10pt}
\end{wrapfigure}

In Fig.~\ref{dais_architecture}, the {\em behavioral programming} abstraction in the handheld component (user accessible clients such as mobile devices) allows application developers to define interaction environments and data requests specification. The {\em virtual sensor} abstraction, on the other hand, employs mechanisms for aggregating heterogeneous data from a variety of physical sensors. 

By providing abstractions to meet resource constraints in immersive sensor networks, DAIS aims to tackle challenges including locality of interactions, mobility induced dynamics, unpredictability of coordination, and complexity of programming\cite{dais}. In particular, a running application needs to efficiently target its interactions only with necessary peers. 

A {\em scene}, a set of declarative specifications for local interactions. encapsulates the node exposure according to property constraints such as hosts, links, and data, providing programmers an abstraction from implicit implementation details. To parameterize a scene, programmers can use {\em metric} to define networking costs, {\em aggregator} to calculate network link, and {\em threshold} to filter the qualified members.

\begin{wrapfigure}{r}{0.6\textwidth}
\vspace{-20pt}
  \begin{center}
    \includegraphics[width=0.6\textwidth]{resources/dais_objects.png}
  \end{center}
  \vspace{-20pt}
  \caption{DAIS Object Diagram\cite{dais} \label{dais_objects}}
    \vspace{-10pt}
\end{wrapfigure}

In Fig.~\ref{dais_objects}, a simplified object flow diagram illustrates how the strategy pattern\cite{gamma1993design} is used to determine how a {\em Query} can be disseminated to a {\em Scene} according to a {\em SceneStrategy} algorithm. In \cite{dais}, only one simple strategy called the {\em BasicScene} was implemented to prototype the proposed middleware. 

With the introduction of the {\em scene}, DAIS encapsulates a localized and adaptive perspective to applications in pervasive computing networks that have highly dynamic mobility and connectivity. It provides programmers with a high-level abstraction interface to build complicated software without worrying about details of sensor context and their interactions.
	
\section{Future Improvements}
With the growing popularity of IoT applications, a significant amount of sensitive data will be pervasive within its PNetS in the form of cached or archived data. Therefore, the balance between encryption and computation overhead should be easily configurable in the design of middleware solutions. Namely, a user should be able to specify how secure her data should be encrypted within the networks, opposed to how much performance sacrifice can be tolerated. 

Given the nature of highly dynamic context and connectivity, data structure employed by IoT middleware needs to have adaptive garbage collectors and distributed caching as well. For example, an archive of historical information may be needed to make prediction model for localized planning and recommendation. In such system, it could be challenging to determine how long for a node to persist its data, if snapshots need to be stored in a regional server, and how a cached data could be routed to exploit the largest throughput.

Finally, considering a sensor networks as a transmission system for signal pulses; it would be interesting to develop a self-adaptive networking protocol that can relocate communication and computing resources according to the query demands in real-time, just like the human neural networks. For example, if certain types of queries are in demand in a scene, busy nodes should be able to make surrounding idle nodes as their replicas to balance the incoming loads, while the system keeps a memory of such demand patterns to reconstruct itself accordingly.

\Section{Conclusion}

To summarize, in this paper we discussed about some of the trending issues in the Internet of Things, three sample middleware solutions addressing different challenges, and in what directions IoT may be facing in the near future. More specifically, we talked about how Grapevine efficiently defines sharing context and group memberships, how Gander employs an unconventional search indexing by sampling query results in PNets, and how DAIS provides a convenient abstraction for sensor linkages and interactions. Finally, we brought up three potential directions for future trends in IoT, including configurable security-performance tradeoffs, distributed caching, and an adaptive on-demand networking system.

\section*{Acknowledgment}
This paper wouldn't be possible without the advices and feedback provided by Dr. Christine Julien and The Department of Electrical and Computer Engineering at The University of Texas at Austin.

%\medskip
%\noindent

\bibliography{report}
\end{document}

